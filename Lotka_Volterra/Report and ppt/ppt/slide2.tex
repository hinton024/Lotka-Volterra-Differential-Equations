







\begin{frame}
    \frametitle{Predator-Prey (Lotka-Volterra) Model}
 %     \framesubtitle{Subtitle 1}
Lotka-Volterra Model: Let $ f ( x , y ) $ be the population of snowshoe hares and $g ( x , y ) $
be the population of lynx..    
    \begin{itemize}
    \item 
     Alfred Lotka (1920), an American biologist and actuary,
published the mathematical predator-prey model and its
cyclical nature.
        \item  The rate of change in a population is equal to the net increase (births) into
the population minus the net decrease (deaths) of the population.
        \item Predators have limitless appetite.
        \item During the process, the environment does not change in favour of one species and the
genetic adaptation is sufficiently slow.
        \item 
        \item The growth model for the hare population is:
\[\frac { d x } { d t } = \alpha x - \beta x y \]
    \end{itemize} 
\end{frame}

    %%%%%%%%%%%%%%%%%%%%%%%%  
