







\begin{frame}
    \frametitle{Predator-Prey (Lotka-Volterra) Model}
 %     \framesubtitle{Subtitle 1}
Lotka-Volterra Model with $ f ( x , y ) $ as the population of snowshoe
hares and $ g ( x , y ) $ as the population of lynx.   
    \begin{itemize}
    \item 
     Alfred Lotka (1920), an American biologist and actuary,
published the mathematical predator-prey model and its
cyclical nature.
        \item  The primary growth for the lynx population depends on
sufficient food for raising lynx kittens, which implies an adequate
nutrients from predation on hares.
        \item This growth rate is similar to the death rate for the hare
population with a different constant of proportionality,
$\delta x y $.
        \item In the absence of hares, the lynx population declines in
proportion to its own population, $ - \gamma y $.
        \item The growth model for the hare population is:
\[ \frac { d y } { d t } = \delta x y - \gamma y \]
    \end{itemize} 
\end{frame}

    %%%%%%%%%%%%%%%%%%%%%%%%  
